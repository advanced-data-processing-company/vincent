\href{https://app.codacy.com/app/maidamai0/vincent?utm_source=github.com&utm_medium=referral&utm_content=advanced-data-processing-company/vincent&utm_campaign=Badge_Grade_Settings}{\tt } \href{https://travis-ci.org/advanced-data-processing-company/vincent}{\tt } \href{https://ci.appveyor.com/project/maidamai0/vincent}{\tt } \href{https://lgtm.com/projects/g/advanced-data-processing-company/vincent/alerts/}{\tt } \href{https://lgtm.com/projects/g/advanced-data-processing-company/vincent/context:javascript}{\tt }

\subsection*{Sql design}


\begin{DoxyItemize}
\item Online

\href{https://docs.google.com/document/d/1QlFliN9hr0bRWUpu1whWGgTl2qbUMroOrxDV7gtdL20/edit}{\tt sql design google document}
\item Offline see \href{https://htmlpreview.github.io/?https://github.com/advanced-data-processing-company/vincent/blob/master/sql_table_design.html}{\tt sql\+\_\+table\+\_\+design} if you can\textquotesingle{}t access the online version.\+Note that this document may be late than online version.
\end{DoxyItemize}

\subsection*{Source structure}

All source files are in {\ttfamily src} directory, each module has its own sub-\/directory in {\ttfamily src}

\subsection*{Build}


\begin{DoxyEnumerate}
\item {\ttfamily cd \$\{V\+I\+N\+C\+E\+N\+T\+\_\+\+R\+O\+O\+T\+\_\+\+P\+A\+TH\}}
\item {\ttfamily mkdir build \&\& cd build}
\item {\ttfamily cmake ..}
\item {\ttfamily make} or {\ttfamily make test} if you want to excute tests
\end{DoxyEnumerate}

{\bfseries D\+O\+NT} build in {\ttfamily src} directory
\begin{DoxyItemize}
\item If you dont want to build {\ttfamily doxygen docs} call cmake with
\end{DoxyItemize}


\begin{DoxyCode}
cmake -DBUILD\_DOC=OFF ..
\end{DoxyCode}


\subsection*{Git commit message}

\subsubsection*{Reasons for these conventions}


\begin{DoxyItemize}
\item automatic generating changelog
\item simple navigation through git history
\end{DoxyItemize}

\subsubsection*{Format of git commit message}


\begin{DoxyCode}
<type>(<scope>): <subject>
# blank line
<body>
# blank line
<footer>
\end{DoxyCode}


\subsubsection*{Example commit message}


\begin{DoxyCode}
fix(app): avoid memory leak

use smart pointer to avoid memory leak

#2310
\end{DoxyCode}


\subsubsection*{first line}

The first line cannot be longer than 70 characters, the second line is always blank and other lines should be wrapped at 80 characters. The type and scope should always be lowercase as shown below.

\paragraph*{Allowed $<$type$>$ values}


\begin{DoxyItemize}
\item feat (new feature for the user, not a new feature for build script)
\item fix (bug fix for the user, not a fix to a build script)
\item docs (changes to the documentation)
\item style (formatting, missing semi colons, etc; no -\/ production code change)
\item refactor (refactoring production code, eg. renaming a variable)
\item test (adding missing tests, refactoring tests; no production code change)
\item chore (updating grunt tasks etc; no production code change)
\end{DoxyItemize}

\paragraph*{$<$scope$>$}

Generally {\ttfamily $<$scope$>$} value should be the module name, such as {\ttfamily app}

\paragraph*{$<$subject$>$}

Briefly concludes modifications

\subsubsection*{Message body}


\begin{DoxyItemize}
\item uses the imperative, present tense\+: “change” not “changed” nor “changes”
\item includes motivation for the change and contrasts with previous behavior
\end{DoxyItemize}

\subsubsection*{Footter}

issue or feature number

\subsubsection*{Config git to use {\ttfamily commit.\+template}}

{\ttfamily git config -\/-\/global commit.\+template ./.gitcommitstyle.\+txt}

\subsubsection*{Reference}


\begin{DoxyItemize}
\item \href{http://karma-runner.github.io/4.0/dev/git-commit-msg.html}{\tt karam git commit style}
\item \href{https://git-scm.com/book/en/v2/Customizing-Git-Git-Configuration}{\tt git config}
\end{DoxyItemize}

\subsection*{Unit test}

every library should has a executable to do test, test excutable name should end with library name plus {\ttfamily \+\_\+test}

take {\ttfamily proto} for example, its test executable is {\ttfamily proto\+\_\+test},when built you can run test with


\begin{DoxyCode}
cd build
bin/proto\_test
\end{DoxyCode}


and it should give something like


\begin{DoxyCode}
Running main() from /home/thy/Documents/vincent/src/3rdparty/src/gtest\_main.cc
[==========] Running 2 tests from 1 test case.
[----------] Global test environment set-up.
[----------] 2 tests from json
[ RUN      ] json.to\_json
[       OK ] json.to\_json (0 ms)
[ RUN      ] json.from\_json
[       OK ] json.from\_json (0 ms)
[----------] 2 tests from json (0 ms total)

[----------] Global test environment tear-down
[==========] 2 tests from 1 test case ran. (0 ms total)
[  PASSED  ] 2 tests.
\end{DoxyCode}


{\ttfamily make \&\& make test}


\begin{DoxyCode}
Running tests...
/usr/bin/ctest --force-new-ctest-process 
Test project /home/thy/Documents/vincent/build
    Start 1: proto
1/2 Test #1: proto ............................   Passed    0.00 sec
    Start 2: log
2/2 Test #2: log ..............................   Passed    0.00 sec

100% tests passed, 0 tests failed out of 2

Total Test time (real) =   0.01 sec
\end{DoxyCode}


\subsection*{How to add a new module}


\begin{DoxyItemize}
\item every module should have a directory in {\ttfamily src}
\item erery module should have a test executable
\item test source files shoudl end with {\ttfamily \+\_\+test}
\item module {\ttfamily C\+Make\+Lists.\+tst} template
\item add {\ttfamily add\+\_\+subdirectory(module\+\_\+name)} in {\ttfamily src/\+C\+Make\+Lists.\+tst}
\item {\bfseries Don\textquotesingle{}t} add any new include indrectory 
\end{DoxyItemize}